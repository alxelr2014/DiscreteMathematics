\chapter{Logic}
\section{Proposition Logic}
\thispagestyle{headings}
A logical \textbf{proposition} is a statement that is either true or false.
\section{Operators}
\begin{definition} [Negation]
Let \(p\) be a proposition, then its negation \(\neg p\), read as ``not \(p\)'', has the opposite truth value of \(p\).
\begin{center}
    \begin{tabular}[c]{c | c}
        \(p\) & \(\neg p\) \\ \hline
        T     & F          \\
        F     & T
    \end{tabular}
\end{center}
\end {definition}

\begin{definition}[Conjunction]
    Let \(p\) and \(q\) be two propositions, then their conjunction \(p \land q\), read as ``\(p\) and \(q\)'', is true only when both \(p\) and \(q\) are true, it is false otherwise.
    \begin{center}
        \begin{tabular}[c]{c  c | c}
            \(p\) & \(q\) & \(p \land q\) \\ \hline
            F     & F     & F             \\
            F     & T     & F             \\
            T     & F     & F             \\
            T     & T     & T             \\
        \end{tabular}
    \end{center}
\end{definition}

Conjunction and not operators are \textbf{universal}, also called \textbf{functionally complete}, in the sense that every boolean function \(f: B^n \to B\) can be generated with these two operators. Note that, every boolean function \(f : B^n \to B\) for \(n \geq 2\) can be generated with binary boolean functions thus, it only suffice to show that every binary and unary boolean function can be generated with \(\{\land, \neg\}\).


\begin{definition}[Disjunction]
    Let \(p\) and \(q\) be two propositions, then their disjunction \(p \lor q\), read as ``\(p\) or \(q\)'', is false only when both \(p\) and \(q\) are false, and true otherwise.
    \begin{center}
        \begin{tabular}[c]{c  c | c}
            \(p\) & \(q\) & \(p \lor q\) \\ \hline
            F     & F     & F            \\
            F     & T     & T            \\
            T     & F     & T            \\
            T     & T     & T            \\
        \end{tabular}
    \end{center}
\end{definition}

Similary, \(\{\lor, \neg\}\) is functionally complete.

\begin{definition}[Exclusive or]
    Let \(p\) and \(q\) be two propositions, then their exclusive or \(p \xor q\), read as ``\(p\) xor \(q\)'', is true only when exactly one the \(p\) or \(q\) is true, it is false otherwise.
    \begin{center}
        \begin{tabular}[c]{c  c | c}
            \(p\) & \(q\) & \(p \xor q\) \\ \hline
            F     & F     & F            \\
            F     & T     & T            \\
            T     & F     & T            \\
            T     & T     & F            \\
        \end{tabular}
    \end{center}
\end{definition}

Curiously, \(\{\xor, \neg\}\) is not universal.

\begin{definition}
    A \textbf{tautology} is a compound proposition that is true no matter what the truth values of its atomic proposition are. A \textbf{contradiction} is a compound proposition that is false no mattter what truth values of its atomic proposition are.
\end{definition}

\begin{definition}
    Compound proposition \(p\) is \textbf{logically equivalent} to \(q\), denoted by \(p \Leftrightarrow q\), when \(p \leftrightarrow q\) is a tautology.
\end{definition}

\section{Predicate Logic}
\textbf{Predicate logic} is an extention of propositional logic. A predicate is a statement that may be true or false depending on the value of its variable.
The collection of value that a variable \(x\) can take is called \(x\)'s \textbf{universe of discourse}.
\textbf{Quantifiers} allow us to quantify how many objects in the universe of discourse satisfy a given predicate.

\begin{definition}
    \textbf{Universal quantifier} \(\forall\) asserts that for all \(x\) in the universe of discourse the predicate is true.
\end{definition}

\begin{definition}
    \textbf{Existential quantifire} \(\exists\) asserts that there exists a \(x\) in the universe of discourse that the predicate is true for that \(x\).
\end{definition}

%proofs
%sets
%functions
%summation

